% ----------------------------------------------------------------
% AMS-LaTeX Paper ************************************************
% **** -----------------------------------------------------------
\documentclass{amsart}
\usepackage{graphicx}
\usepackage{fix-cm}
\usepackage{amsmath, gauss}
\usepackage{etoolbox}
\usepackage{amssymb}
\usepackage{bm}
\usepackage{circuitikz}
% This makes a command for Christoffel-Symbols
\usepackage{relsize}
\newcommand{\Christoffel}[2]{\ensuremath{{\mathlarger{\mathlarger\Gamma}}^{#1}\!\!\!_{#2}}}

\begin{document}

% \title{Neural Computer\\{\bf Final Project}}%
% \author{Renan Monteiro Barbosa}%
% \date{05/16/2024}

% \maketitle
% \clearpage

\thispagestyle{plain}
\begin{center}
    \Large
    \textbf{Neural Computer}
        
    \vspace{0.4cm}
    \large
    Thesis Subtitle
        
    \vspace{0.4cm}
    \textbf{Renan Monteiro Barbosa}
       
    \vspace{0.9cm}
    \textbf{Abstract}
    
    Lorem ipsum dolor sit amet, consectetur adipiscing elit, sed do eiusmod tempor incididunt ut labore et dolore magna aliqua. 
    Turpis egestas pretium aenean pharetra magna ac placerat vestibulum. 
    Morbi non arcu risus quis. Vel turpis nunc eget lorem dolor sed viverra. 
    Arcu cursus euismod quis viverra nibh cras pulvinar mattis. 
    Pretium fusce id velit ut tortor pretium viverra suspendisse potenti. 
    Pellentesque diam volutpat commodo sed egestas egestas fringilla. 
    Semper risus in hendrerit gravida. 
    Est placerat in egestas erat imperdiet sed euismod nisi porta. 
    Nunc vel risus commodo viverra maecenas accumsan lacus vel facilisis. 
    Cras pulvinar mattis nunc sed blandit libero volutpat sed. 
    Accumsan in nisl nisi scelerisque eu ultrices vitae auctor eu. 
    At imperdiet dui accumsan sit amet nulla facilisi. 
    Tempus quam pellentesque nec nam aliquam sem et tortor. 
    Quam elementum pulvinar etiam non quam. 
    Pretium aenean pharetra magna ac placerat vestibulum lectus mauris ultrices. 
    Nunc aliquet bibendum enim facilisis. 
    Lorem mollis aliquam ut porttitor leo a diam sollicitudin. 
    Adipiscing elit pellentesque habitant morbi. 
    Feugiat sed lectus vestibulum mattis ullamcorper velit sed ullamcorper morbi.
\end{center}

\clearpage
% Reference to Shagrir Begin
% Shagrir 2022 p 240

Use Shagrir book The Nature of Physical Computing to formalize Computing as Modelling.


Sagrir proposes that computing is satisfied with modeling of the input-output type with some degree of morphism.

Assuming that computing is a process of the physical system that transforms (physical) input variables into output variables, the mirroring condition is as follows:



% Below is found at page 240

A physical system P is a computing system just in case:

\begin{enumerate}
    % ---------------------------------------------------------------------------------
    % 1-Input-Output Mirroring
    % ---------------------------------------------------------------------------------
    \item \textbf{Input-Output Mirroring}. The input-output function, g, of a given process in \textbf{P} preserves a certain relation, \underline{R}, in a target domain \textbf{T}: there is a mapping from \textbf{P} to \textbf{T} that maps g to \underline{R}, x to \underline{x}, y to \underline{y}, \dots , such that $g(x) = y$ iff $\left\langle \underline{x},\underline{y} \right\rangle\ \; \epsilon \; \underline{R}$. This means that g and \underline{R} share some formal relation \textbf{f}.
    % ---------------------------------------------------------------------------------
    % 2-Implementing
    % ---------------------------------------------------------------------------------
    \item \textbf{Implementing}. This process of \textbf{P}, whose input-output function is g, implements some formalism \textbf{S} whose input-output (abstract) function is \textbf{f}.
    % ---------------------------------------------------------------------------------
    % 3-Representing
    % ---------------------------------------------------------------------------------
    \item \textbf{Representing}. The input variables x of \textbf{P} represent the entities \underline{x} of \textbf{T}, and the output variables y of \textbf{P} represent the entities \underline{y} of \textbf{T}.
    % ----------------------------------------------------------------
\end{enumerate}

The underlined italicized symbols (such as x and y) to signify properties of the target domain.

% Reference to Shagrir End

% Reference to Neural Manifolds — Linear Algebra and Topology in Neuroscience Begin
% https://medium.com/bits-and-neurons/neural-manifolds-linear-algebra-and-topology-in-neuroscience-dde5a8181811

% https://www.nature.com/articles/nature13665

The paper Sadtler et. al. on Neural Manifolds cleverly establishes that neural activity is inherently constrained by properties of the physical network circuitry itself.

These constraints result in neural activity patterns that comprise a low-dimensional subspace — the manifold — within the larger possible high-dimensional neural space.

The authors relate this discovery to skill learning and adaptation

% Reference to Neural Manifolds — Linear Algebra and Topology in Neuroscience End

\clearpage

References

% Below References are found on the video
% Your brain is moving along the surface of the torus - https://www.youtube.com/watch?v=9ujnZcaqf-4


% Gardner, R. J. et al. Toroidal topology of population activity in grid cells. Nature 602, 123–128 (2022).
% Pisokas, I., Heinze, S. & Webb, B. The head direction circuit of two insect species. eLife 9, e53985 (2020). https://elifesciences.org/articles/53985
% Shilnikov, A. L. & Maurer, A. P. The Art of Grid Fields: Geometry of Neuronal Time. Front. Neural Circuits 10, (2016).
% Moser, M.-B., Rowland, D. C. & Moser, E. I. Place Cells, Grid Cells, and Memory. Cold Spring Harb Perspect Biol 7, a021808 (2015).
% Lewis, M., Purdy, S., Ahmad, S. & Hawkins, J. Locations in the Neocortex: A Theory of Sensorimotor Object Recognition Using Cortical Grid Cells. https://www.biorxiv.org/content/10.1101/436352v2 (2018) doi:10.1101/436352.

% Mentioned by GUS on the TEMt
% Connectome of the fly visual circuitry. - https://www.janelia.org/publication/connectome-fly-visual-circuitry

% This is Referenced as the Function u(x)
% Mathematical equivalence of two common forms of firing rate models of neural networks
% https://pubmed.ncbi.nlm.nih.gov/22023194/

\clearpage

Important terms

metric tensor

Levi-Civita connection

Affine Connection

Covariant Derivative

Christoffel symbols $ \Christoffel{i}{jk} $

\end{document}
% ----------------------------------------------------------------
