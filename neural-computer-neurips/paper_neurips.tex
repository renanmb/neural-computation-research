\documentclass{article}

% if you need to pass options to natbib, use, e.g.:
%     \PassOptionsToPackage{numbers, compress}{natbib}
% before loading neurips_2019

% ready for submission
% \usepackage{neurips_2019}

% to compile a preprint version, e.g., for submission to arXiv, add add the
% [preprint] option:
%     \usepackage[preprint]{neurips_2019}

% to compile a camera-ready version, add the [final] option, e.g.:
\usepackage[nonatbib, final]{neurips_2019}


% to avoid loading the natbib package, add option nonatbib:
%     \usepackage[nonatbib]{neurips_2019}

\usepackage[utf8]{inputenc} % allow utf-8 input
\usepackage[T1]{fontenc}    % use 8-bit T1 fonts
\usepackage{hyperref}       % hyperlinks
\usepackage{url}            % simple URL typesetting
\usepackage{booktabs}       % professional-quality tables
\usepackage{amsfonts}       % blackboard math symbols
\usepackage{nicefrac}       % compact symbols for 1/2, etc.
\usepackage{microtype}      % microtypography

\usepackage{float}
\usepackage{graphicx}
\usepackage{subfig}

\usepackage{cite}


\title{TEM-t as The Universal Neural Computer}

% The \author macro works with any number of authors. There are two commands
% used to separate the names and addresses of multiple authors: \And and \AND.
%
% Using \And between authors leaves it to LaTeX to determine where to break the
% lines. Using \AND forces a line break at that point. So, if LaTeX puts 3 of 4
% authors names on the first line, and the last on the second line, try using
% \AND instead of \And before the third author name.

\author{
  \textbf{Renan Monteiro Barbosa$^{1}$} \\ 
  $^{1}$University of West Florida \\ 
  \texttt{rmb54f@students.uwd.edu, \{email1, email2,\}@affiliation.edu}
}

\begin{document}

\maketitle

% #########################################################################################
% #########################################################################################
% Abstract
% #########################################################################################
% #########################################################################################


\begin{abstract}
The abstract paragraph should be indented \nicefrac{1}{2}~inch (3~picas) on both the left- and right-hand margins. Use 10~point type, with a vertical spacing (leading) of 11~points.  The word \textbf{Abstract} must be centered, bold, and in point size 12. Two line spaces precede the abstract. The abstract must be limited to one paragraph.

Tempus quam pellentesque nec nam aliquam sem et tortor. 
Quam elementum pulvinar etiam non quam. 
Pretium aenean pharetra magna ac placerat vestibulum lectus mauris ultrices. 
Nunc aliquet bibendum enim facilisis. 
Lorem mollis aliquam ut porttitor leo a diam sollicitudin. 
Adipiscing elit pellentesque habitant morbi. 
Feugiat sed lectus vestibulum mattis ullamcorper velit sed ullamcorper morbi.
\end{abstract}

% #########################################################################################
% #########################################################################################
% Introduction
% #########################################################################################
% #########################################################################################

\section{Introduction}

Advancements in the last few decades have brought us to a point at which what was thought to only be theorized by Tolman (1948) as he argued that for Humans and other animals to make complex inferences from sparse observations and rapidly integrate new knowledge to control their behavior there should exist a systematic organization of such knowledge on what could be called a cognitive map. (note, try to make it sound like the theory of computing)

But what was missing from these early studies, was a way to address the neuronal mechanisms that led information to be stored as memory. Further development has shown that place cells are part of a wider network of spatially modulated neurons, including grid, border, and head direction cells, each with distinct roles in the representation of space and spatial memory.

Bringing to surface a mechanistic basis for memory formation (Nobel prize winning Place Cells, Grid Cells, and Memory) which later has been suggested that relational memory and spatial reasoning might be related by a common mechanism (Eichenbaum and Cohen, 2014).


\cite{beiran2023parametric}


(Must do a paper dump showcasing some advancements)
In the last few decades we have seen many advancements in neuroscience. It is now possible to record activity from every neuron in a zebrafish larva’s brain while it is freely swimming and responding to stimuli (Kim et al. 2017). We have a nearly complete map of one hemisphere of the fly brain, with every neuron and most of its synapses accounted for (Pipkin 2020). The neurons responsible for a mouse’s memory of an event can be recorded, tagged, and replayed by laser stimulation, causing the mouse to behave as if the event had happened again (CarrilloReid et al. 2019; Ramirez et al. 2013). (This references can be found on John M Beggs - The Cortex and the Critical Point)

Such advancements brought to light evidence for what was once only a theory, for example, Tolman (1948) argued that for Humans and other animals to make complex inferences from sparse observations and rapidly integrate new knowledge to control their behavior there should exist a systematic organization of such knowledge on what could be called a cognitive map. (note, try to make it sound like the theory of computing - references can be found on the TEM paper)

But what was missing from these early studies, was a way to address the neuronal mechanisms that led information to be stored as memory. Further development has shown that place cells are part of a wider network of spatially modulated neurons, including grid, border, and head direction cells, each with distinct roles in the representation of space and spatial memory.

The combination of all these technological and theoretical developments demonstrated through evidence and theoretical models that there is indeed a mechanistic basis for memory formation, this was shown in the Nobel winning work done by May-Britt Moser, Edvard I. Moser (reference the nobel prize)(Nobel prize winning Place Cells, Grid Cells, and Memory). This work has inspired new theories in theoretical-neuroscience that proposes the idea that relational memory and spatial reasoning might be related by a common mechanism (Eichenbaum and Cohen, 2014 - found in the TEM paper). (need to work on this paragraph)

Limited evidence provided in recent work with grid cells has shown, there is mechanism generating invariant representations, by using simultaneous recordings from many hundreds of grid cells and subsequent topological data analysis that the joint activity of grid cells from an individual module (neuronal population) resides on a toroidal manifold as expected in a two-dimensional CAN (Continuous Attractor Network). The positions are maintained between environments and from wakefulness to sleep, demonstrating to be invariant representations. This research demonstrated ,with some limitations, network dynamics on a toroidal manifold and provided a population-level visualization of CAN dynamics in grid cells.
(reference Toroidal topology of population activity in grid cells)

Even though the limitations 

There is growing evidence to support that manifolds maintain a well-preserved covariance across tasks. These results support the view that complex computation emerges from the flexible activation of different combinations of “Neural modes” (Need to find a better term) which themselves arise from the Network Connectivity (reference to Hopfield is all you need and Transformers) (Cortical population activity within a preserved neural manifold underlies multiple motor behaviors, reference 8 for Emergence of universal computations through neural manifold dynamics)

In this paper we want to show that there exists a morphism between Neuroscience models and existing NN models, so we can approximately represent the Neural Manifolds with the Transformers.

Must note that we are not saying the brain is closely related to transformers, instead we are  using a mathematical relationship between the so popular transformers and the attractor neural networks that have been carefully formulated in neuroscience models as being an essential piece in forming invariant representations in a topological structure that facilitates the emergence of Computation Through Neural Population Dynamics.
Additionally, by using a modified transformers model we can leverage several tools, techniques, software and hardware that have been matured and extensively tested given the popularity and widespread use of transformers. Further studies are required to develop and validate a better architecture that better represents Neural Population Dynamics.



\subsection{Retrieval of style files}

The style files for NeurIPS and other conference information are available on the World Wide Web at  \url{http://www.neurips.cc/}. The file \verb+neurips_2019.pdf+ contains these instructions and illustrates the various formatting requirements your NeurIPS paper must satisfy.

The only supported style file for NeurIPS 2019 is \verb+neurips_2019.sty+, rewritten for \LaTeXe{}. \textbf{Previous style files for \LaTeX{} 2.09,  Microsoft Word, and RTF are no longer supported!}


\subsection{Retrieval of style files}

The style files for NeurIPS and other conference information are available on the World Wide Web at  \url{http://www.neurips.cc/}. The file \verb+neurips_2019.pdf+ contains these instructions and illustrates the various formatting requirements your NeurIPS paper must satisfy.

The only supported style file for NeurIPS 2019 is \verb+neurips_2019.sty+, rewritten for \LaTeXe{}. \textbf{Previous style files for \LaTeX{} 2.09,  Microsoft Word, and RTF are no longer supported!}


\subsection{Retrieval of style files}

The style files for NeurIPS and other conference information are available on the World Wide Web at  \url{http://www.neurips.cc/}. The file \verb+neurips_2019.pdf+ contains these instructions and illustrates the various formatting requirements your NeurIPS paper must satisfy.

The only supported style file for NeurIPS 2019 is \verb+neurips_2019.sty+, rewritten for \LaTeXe{}. \textbf{Previous style files for \LaTeX{} 2.09,  Microsoft Word, and RTF are no longer supported!}


% #########################################################################################
% #########################################################################################
% Theory of Computing
% #########################################################################################
% #########################################################################################

\section{Theory of Computing}

Computing is not a new concept, there is historical evidence that the Greeks were already capable of performing computation to predict the astronomical movements giving a programmable input (Antikythera mechanism).


% #########################################################################################
% #########################################################################################
% Example - Adding a Single big Image
% #########################################################################################
% #########################################################################################

\section{Single image}

Lorem ipsum dolor sit amet, consectetur adipiscing elit, sed eiusmod tempor incidunt ut labore et dolore magna aliqua. Ut enim ad minim veniam, quis nostrud exercitation ullamco laboris nisi ut aliquid ex ea commodi consequat. Quis aute iure reprehenderit in voluptate velit esse cillum dolore eu fugiat nulla pariatur. Excepteur sint obcaecat cupiditat non proident, sunt in culpa qui officia deserunt mollit anim id est laborum. Fig.~\ref{image1} shows the NeurIPS logo.


\begin{figure}[H]
  \centering{\includegraphics[width=40mm]{images/logo-neurips.png}}
  \caption{Example of single image}
  \label{image1}
\end{figure}


% #########################################################################################
% #########################################################################################
% Example - Adding multiple Images
% #########################################################################################
% #########################################################################################

\section{Multiple images}

Lorem ipsum dolor sit amet, consectetur adipiscing elit, sed eiusmod tempor incidunt ut labore et dolore magna aliqua. Ut enim ad minim veniam, quis nostrud exercitation ullamco laboris nisi ut aliquid ex ea commodi consequat. Quis aute iure reprehenderit in voluptate velit esse cillum dolore eu fugiat nulla pariatur. Excepteur sint obcaecat cupiditat non proident, sunt in culpa qui officia deserunt mollit anim id est laborum. Fig.~\ref{image1} shows the NeurIPS logo as well as Fig.~\ref{image2}.


\begin{figure}[H]
\centering
\subfloat[]{\includegraphics[width=40mm]{images/logo-neurips.png}} \quad \quad 
\subfloat[]{\includegraphics[width=40mm]{images/logo-neurips.png}} \quad
\caption{Examples for sub-images} \label{image2}
\end{figure}


\begin{figure}[H]
  \centering
  \begin{minipage}[b]{40mm}
    \includegraphics[width=40mm]{images/logo-neurips.png}
    \caption{Logo image}
    \label{image3}
  \end{minipage} 
  \quad \quad \quad \quad \quad
  \begin{minipage}[b]{40mm}
    \includegraphics[width=40mm]{images/logo-neurips.png}
    \caption{Logo image}
    \label{image4}
  \end{minipage}
\end{figure}


% #########################################################################################
% #########################################################################################
% Example - Citations
% #########################################################################################
% #########################################################################################
\section{Some other Section}

Citations examples used to be here


% #########################################################################################
% #########################################################################################
% Example - Tables
% #########################################################################################
% #########################################################################################
\section{Tables}

Lorem ipsum dolor sit amet, consectetur adipiscing elit, sed eiusmod tempor incidunt ut labore et dolore magna aliqua. Ut enim ad minim veniam, quis nostrud exercitation ullamco laboris nisi ut aliquid ex ea commodi consequat. Quis aute iure reprehenderit in voluptate velit esse cillum dolore eu fugiat nulla pariatur. Excepteur sint obcaecat cupiditat non proident, sunt in culpa qui officia deserunt mollit anim id est laborum. The Table~\ref{sample-table} shows an example of a table.

\begin{table}[H]
  \caption{Sample table title}
  \label{sample-table}
  \centering
  \begin{tabular}{lll}
    \toprule
    \multicolumn{2}{c}{Part}                   \\
    \cmidrule(r){1-2}
    Name     & Description     & Size ($\mu$m) \\
    \midrule
    Dendrite & Input terminal  & $\sim$100     \\
    Axon     & Output terminal & $\sim$10      \\
    Soma     & Cell body       & up to $10^6$  \\
    \bottomrule
  \end{tabular}
\end{table}

% #########################################################################################
% #########################################################################################
% Conclusion
% #########################################################################################
% #########################################################################################
\section{Conclusions}

Lorem ipsum dolor sit amet, consectetur adipiscing elit, sed eiusmod tempor incidunt ut labore et dolore magna aliqua. Ut enim ad minim veniam, quis nostrud exercitation ullamco laboris nisi ut aliquid ex ea commodi consequat. Quis aute iure reprehenderit in voluptate velit esse cillum dolore eu fugiat nulla pariatur. Excepteur sint obcaecat cupiditat non proident, sunt in culpa qui officia deserunt mollit anim id est laborum.


% #########################################################################################
% #########################################################################################
% Acknowledgments
% #########################################################################################
% #########################################################################################
\subsubsection*{Acknowledgments}

Use unnumbered third level headings for the acknowledgments. All acknowledgments go at the end of the paper. Do not include acknowledgments in the anonymized submission, only in the final paper. This example was prepared by Dennis Núñez Fernández.

% #########################################################################################
% #########################################################################################
% References
% #########################################################################################
% #########################################################################################

\bibliography{references}{}
\bibliographystyle{plain}


\end{document}
